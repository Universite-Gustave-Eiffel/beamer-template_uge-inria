%%%% 		UsersPackagesCommands.tex
%%%%		version 0.1.1 du 01/11/2023
%%%% 		Author: Romain NOËL
%%%%		email: romain.noel@univ-eiffel.fr
%%%%--------------------------------------------%%%%

%%%%---- OS & FONT     -------------------------%%%%
%%%%---- POLICE     ----------------------------%%%%
%%%%---- TEXT FORMATTING -----------------------%%%%
	% \usepackage{enumitem}
	% \usepackage{ulem}
		
	%%%% Personal %%%%
	\makeatletter
		\newcommand{\ie}{\@ifstar{\emph{i.e.\@}}{\emph{id est}\ }}
		\newcommand{\eg}{\@ifstar{\emph{e.g.\@}}{\emph{exempli gratia}\ }}
		\newcommand{\cad}{\@ifstar{c.-\`a.-d.\@}{c'est-à-dire}\ } 
		\newcommand{\cf}{\@ifstar{\emph{cf.\@}}{\emph{confer}\ }}
	\makeatother
	\newcommand{\famName}[1]{\textsc{#1}}
	

%%%%---- SPACING & JUSTIFICATION ---------------%%%%
%%%%---- LAYOUT & MARGINS ----------------------%%%%
%%%%---- FOOT PAGE & HEAD ----------------------%%%%
%%%%---- TABLE of CONTENT & FIGURES ------------%%%%
%%%%---- PARTS & CHAPTERS ----------------------%%%%
%%%%---- WATERMARKS ----------------------------%%%%
%%%%---- COVER PAGE ----------------------------%%%%
%%%%---- IMAGES --------------------------------%%%%
	%\usepackage{placeins} 	% provide FloatBarrier
	%\usepackage[absolute,overlay]{textpos} % for positioning text or floats at absolute coords.
	\usepackage{subcaption} 	% multiple figures 
		%\renewcommand{\thesubfigure}{{\normalfont\alph{subfigure}}} %

%%%%---- DRAWING & GRAPHS ----------------------%%%%
	\usepackage{import}
	\usepackage{tikz} 	%
	\usepackage{pgfplots} 	%
		\pgfplotsset{compat=newest} 	%
		\pdfsuppresswarningpagegroup=1


%%%%---- TABLES & BOXES ------------------------%%%%
	% \usepackage{tcolorbox} % colored box with a lot of flexibility
	% 	\tcbuselibrary{skins,theorems} % extra for pseudo(code)
	\usepackage{tabularray} % Typeset tabulars and arrays (contains equivalent of longtable, booktabs and dcolumn at least) 
		\UseTblrLibrary{booktabs} % to load extra commands from booktabs
		% \UseTblrLibrary{diagbox} % for cells with a diagbox
		% \UseTblrLibrary{varwidth} % for cell with variable width
		% \UseTblrLibrary{amsmath} % to improve with +array, +matrix, +bmatrix, +Bmatrix, +pmatrix, +vmatrix, +Vmatrix and +cases
		% \UseTblrLibrary{siunitx} % for cell using SI formatting


%%%%---- SYMBOLS -------------------------------%%%%
%%%%---- MATHS & PHYSICS -----------------------%%%%
	\usepackage{amsmath, amsfonts, amssymb} % American Math Standard
	\usepackage{mathtools} % extension of AMS with tools
	\usepackage{mleftright} % better left and right bracket management
	\usepackage{derivative} % standard and customable derivative operators
	\usepackage{siunitx} % International System units
	% \usepackage{tensors} % work in progress ^^
	% \usepackage{empheq} % emphasize equations
	\usepackage[amsthm, amsmath, thmmarks, hyperref]{ntheorem}%[standard]
		\newtheorem{proposition}{Proposition}
		\newtheorem{defs}{Definition}
		\newtheorem{theorem}{Theorem}
		% \newtheorem{proof}{Proof}
	% \usepackage[ntheorem]{empheq} 	% emphasize equations


%%%%---- THEOREMS ------------------------------%%%%
%%%%---- CHEMISTRY & BIO -----------------------%%%%
%%%%---- CODING, LINKS & OTHERS ----------------%%%%
	% \usepackage{listings} % to include piece of codes
	% \usepackage{pseudo} % for flexible pseudocode
	% 	\newtcbtheorem{algorithm}{Algorithm}{pseudo/booktabs, float}{alg} % create a float env for algoritms
	\usepackage{cleveref}%
	\usepackage{standalone} 	% Other include compilable alone.
		
		
%%%%---- INDEX & GLOSSARIES --------------------%%%%
	\usepackage[record, automake, section=section]{glossaries-extra}
		%Newglossaries
		\newglossary*{acronym}{Abbreviations}
		\newglossary*{notation}{Notation}
		\newglossary*{symbol}{Nomenclature}
		
		% new keys must be defined before \GlsXtrLoadResources    
		\glsaddstoragekey{unit}{}{\glsentryunit}
		
		% define style of the abbreviation in the text
		\setabbreviationstyle[acronym]{long-short}
		\setabbreviationstyle[foreignabbreviation]{long-short-user}
		\setabbreviationstyle[inline]{long-short}	% 
		\setabbreviationstyle[footer]{footnote} 	% 
		
		
		% \newcommand{\AtEndPreambPerso}{
		\GlsXtrLoadResources[
			src={\PathSymbolsEntries}, % entries in symbols.bib
			%selection=all,
			%save-locations=false,% location list not needed
			entry-type-aliases={main=entry, notation=entry, foreignabbreviation=abbreviation},
			field-aliases={
				foreignlong=user1,
				nativelong=long
			},
			type=acronym,
			match={entrytype=(acronym|abbreviation)},
			category={same as original entry}% requires bib2gls v1.4+
			%sort
		]

		\GlsXtrLoadResources[
			src={\PathSymbolsEntries}, % entries in symbols.bib
			%selection=all,
			%save-locations=false,% location list not needed
			entry-type-aliases={main=entry, notation=entry, foreignabbreviation=abbreviation},
			type=symbol,
			match={entrytype=symbol},
			sort-field={category},% sort @symbol entries by category field
			symbol-sort-fallback=description,% sort @symbol entries by name field
			identical-sort-action=name,% sort by description if sort value identical
			sort={letter-nocase},
			% symbol-sort-fallback=name,% sort @symbol entries by name field
			% identical-sort-action=description,% sort by description if sort value identical
		]
		% }
		\glsdisablehyper


%%%%---- BIBLIOGRAPHY --------------------------%%%%
	\usepackage[%citestyle=numeric,  	% style de citation : alphabetic, numeric, 		authoryear,
		%bibstyle=numeric, 	% style de la biblio : alphabetic, authoryear, numeric,
		style=numeric,			% style gloabel + apa, ieee, ams : style predefinit par certains organisme
		sorting=none,
		mcite=true,				% collapsing  multiple  citations into one
		subentry,				% have entry that referes to another (not with authoryear)
		labelnumber, 			%
		citetracker=true,		% To track citations
		% backref=true,			% afficher les pages ou les citations sont dans la biblio
		url=false,
		giveninits=true,   	% Initials for first name
		maxcitenames=3,			% nb max de noms lors de \cite, audela "et al."
		maxbibnames=100,		% nb max de noms dans la biblio, audela "et al."
		defernumbers=true,		% ne pas charger toute la base de donnees, uniquement le necessaire
		backend=biber			% bibtex, biber
	]{biblatex}

		% \AtEveryBibitem{\clearfield{url}}
		\renewcommand{\mkbibnamefamily}[1]{\textsc{#1}}

		\addbibresource[datatype=bibtex]{\Pathbasebib}
		
		% Create a \defbibentrysetlabel command in order to redefine the printing of mcite manually
		\makeatletter
			% warning the file called \blxmset@bibfile@name will be
			% overwritten without warning
			\def\blxmset@bibfile@name{\jobname -msets.bib}
			\newwrite\blxmset@bibfile
			\immediate\openout\blxmset@bibfile=\blxmset@bibfile@name
			\immediate\write\blxmset@bibfile{%
				@comment{auxiliary file for \string\defbibentrysetlabel}^^J%
				@comment{This file may safely be deleted.
					It will be recreated as required.}}
			
			\AtEndDocument{%
				\closeout\blxmset@bibfile}
			
			\newrobustcmd*{\defbibentrysetlabel}[3]{%
				\@bsphack
				\immediate\write\blxmset@bibfile{%
					@set{#1, entryset = {\unexpanded{#3}}, %
						shorthand = {\unexpanded{#2}},}%
				}%
				\nocite{#1}%
				\@esphack}
			
			\addbibresource{\blxmset@bibfile@name}
		\makeatother
		
		\renewcommand{\entrysetpunct}{.\newline}
		
		% Change the footnote style with letters to avoid confusion.
		\renewcommand{\thefootnote}{\alph{footnote}}
		% Change the style of the printing biblio to let appear the numbers like in the footnote
		\setbeamertemplate{bibliography item}{\insertbiblabel}
		
		% Define the new cite command using the footnote
		\DeclareCiteCommand{\footfullciteBeamer}
		{\usebibmacro{prenote}} 
		{%
			\renewcommand{\thefootnote}{\arabic{footnote}}% Switch to footnote with numbers
			\footnotemark[\thefield{labelnumber}]% Add the mark corresponding to the number entry% 
			\footnotetext[\thefield{labelnumber}]{%  Add the footnote text with same number entry.
				% \printfield{labelnumber}
				\printnames{labelname}% The name 
				\setunit{\printdelim{nametitledelim}}% separator
				\printfield[citetitle]{labeltitle}% The title
				\setunit{\addperiod\space}% separator
				\printfield{year}% The year
			}%
			\renewcommand{\thefootnote}{\alph{footnote}}% Switch back to footnote with letters.
		}
		{\multicitedelim}
		{\usebibmacro{postnote}}
		
		%
		\DeclareCiteCommand{\fullciteBeamer}
		{\usebibmacro{prenote}}
		{\printnames{labelname}%
			\setunit{\printdelim{nametitledelim}}%
			\printfield[citetitle]{labeltitle}%
			\setunit{\addperiod\space}%
			\printfield{year}%
		}
		{\multicitedelim}
		{\usebibmacro{postnote}}
		
		
%%%%---- MULTIMEDIA ---------------------------%%%%
	\usepackage{animate} % Create embedded animation from images on all OS and viewer !

		% Os variables for the macros
		%\def\OSvar{linux} % windows, mac, linux
		\def\OSlinux{linux}
		\def\OSwindows{windows}
		\def\OSmac{mac}
		\providecommand{\includeVideo}[3][]{The command includeVideo is not correctly defined for your OS.}

	%%%%%%%%%%%%%%%%%%%%%%%%%%%%%%%%%%%%%%%%
	% To convert Video with right codec !!!!
	%%%%%%%%%%%%%%%%%%%%%%%%%%%%%%%%%%%%%%%%
	% ffmpeg -i Bubble_light.avi -vf scale="trunc(iw/2)*2:trunc(ih/2)*2" -c:v libx264 -profile:v high -pix_fmt yuv420p -g 30 -r 30 Bubble_light.mp4

		% Load the useful package for linux
		\ifx\OSvar\OSlinux
			\usepackage{multimedia}
			
			\renewcommand{\includeVideo}[3][]{%
				\movie[ % On linux with okular ++ poppler and phonon-backend-vlc installed
					showcontrols=true, %
					%poster, %
					#1% some options like size
				]%
				{#3}%
				{#2}
			}
		\fi
		% for windows
		\ifx\OSvar\OSwindows
			\usepackage{media9}
			
			\renewcommand{\includeVideo}[3][]{%
				\includemedia[% % Windows AcrobatReader >9.1
					%	activate=pagevisible,%
					%   activate=onclick, %this is default
					%	deactivate=pageclose,%
					addresource=#2,%
					flashvars={%
						src=#2 % same path as in addresource!
						%	&autoPlay=true % default: false; if =true, automatically starts playback after activation (see option ‘activation)’
						%	&loop=true % if loop=true, media is played in a loop
						%	&controlBarAutoHideTimeout=0 %  time span before auto-hide
					},%
					#1% some options like size%
				]{#3}{StrobeMediaPlayback.swf}% %{SlideShow.swf} %{APlayer.swf} %{StrobeMediaPlayback.swf} %{VPlayer.swf}
			}% end of the new command
		\fi
		

%%%%---- TRANSLATION ---------------------------%%%%
%%%%---- VARIABLES -----------------------------%%%%
%%%%---- MACROS --------------------------------%%%%
%%%%---- HYPHENATIONS & ACRONYMS --------------%%%%

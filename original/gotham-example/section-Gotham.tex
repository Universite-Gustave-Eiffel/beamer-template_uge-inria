\documentclass[aspectratio=169]{beamer}
\usetheme{gotham}

	\usepackage{standalone}
	\usepackage{tikz}
	\usepackage{pgfplots}
	\usepackage{tabularray} % Typeset tabulars and arrays (contains equivalent of longtable, booktabs and dcolumn at least) 
		\UseTblrLibrary{booktabs} % to load extra commands from booktabs
	\usepackage{changepage}

	\newcommand{\themename}{\textbf{\textsc{Gotham}}}


\begin{document} 

\section{Gotham Theme}

	% FRAME
	\begin{frame}[fragile]{Gotham}
	
		The \themename{} theme is a Beamer theme with a minimal-ish visual style largely inspired by the \href{https://github.com/matze/mtheme}{\textsc{Metropolis} Beamer Theme} by Matthias Vogelgesang (and some other Beamer themes).

		Yet, \themename{} is highly extendable and versatile.
		\bigskip
		
		First, enable the theme by classically loading it:
		
		\begin{verbatim}
			\documentclass{beamer}
			\usetheme{gotham}
		\end{verbatim}
		
		Then, the all customization can be performed at any moment in the presentation using:

		\begin{verbatim}
			\gothamset{<option>=...}
		\end{verbatim}
	\end{frame}


\subsection{Fonts}

	% FRAME	
	\begin{frame}[fragile]{Gotham title formats}
		Note, that you have to have Mozilla's \emph{Fira Sans} font and XeTeX or LuaTeX installed to enjoy this wonderful typography.

		\begin{columns}[T,onlytextwidth]
		\column{0.49\textwidth}
			\themename{} supports 4 different title formats \verb|\gothamset{format frametitle=}|
			\begin{itemize}
				\item regular
				\item \MakeLowercase{Lower}
				\item \MakeUppercase{Upper}
				\item \MakeTitlecase{Title Case}
			\end{itemize}
		\column{0.49\textwidth}
			\themename{} supports 3 different title shape \verb|\gothamset{shape frametitle=...}|:
			\begin{itemize}
				\item regular
				\item \textsc{Small caps}
				\item \textit{italic}
			\end{itemize}
		\end{columns}
		
		\vspace{2em}
		They can either be set at once for every title type or individually.
	\end{frame}

	{ \gothamset{shape frametitle=smallcaps, format frametitle=titlecase}
	% FRAME
	\begin{frame}{Titles: Small caps and titlecase}
		This frame uses the title format options: \texttt{shape frametitle=smallcaps, format frametitle=titlecase}.

		\begin{alertblock}{Potential Problems}
			Be aware that not every font supports small caps. 
			If for example you typeset your presentation with pdfTeX and the Computer Modern Sans Serif font, every text in small caps will be typeset with the Computer Modern Serif font instead.
			Please refer to the documentation if you consider using it.

			As a rule of thumb: just use it for plaintext-only titles.
		\end{alertblock}
	\end{frame}
	}

	{ \gothamset{format frametitle=upper, shape frametitle=italic}
	% FRAME
	\begin{frame}{Titles: Upper and italic}
		This frame uses the title format options: \texttt{format frametitle=upper, shape frametitle=smallcaps}.

		\begin{alertblock}{Potential problems}
			As this title format also uses small caps you face the same problems as with the \texttt{smallcaps} title format. 
			Additionally this format can cause some other problems. 
			Please refer to the documentation if you consider using it.
		\end{alertblock}
	\end{frame}
	}

	{ \gothamset{format frametitle=lower}
	% FRAME
	\begin{frame}{Titles: LOWER and regular}
		This frame uses the title format options: \texttt{format frametitle=lower, shape frametitle=regular}.
	\end{frame}
	}


\subsection{Colors}

	{ \gothamset{background=dark}
	% FRAME
	\begin{frame}[fragile]{Presentation style via background color}
		The color mode (a.k.a. background color) can be changed using:
		\begin{verbatim} \gothamset{background=dark | light | transparent} \end{verbatim}
	\end{frame}
	}

	% FRAME	
	\begin{frame}[fragile]{Blocks}
		Three different block environments are pre-defined and may be styled with an optional background color.
		
		\begin{columns}[T,onlytextwidth]
		 \column{0.3\textwidth}
		 	\begin{verbatim}\gothamset{
				block=native}\end{verbatim}

		   \begin{block}{Default}
		     Block content.
		   \end{block}
		
		   \begin{alertblock}{Alert}
		     Block content.
		   \end{alertblock}
		
		   \begin{exampleblock}{Example}
		     Block content.
		   \end{exampleblock}
		
		 \column{0.3\textwidth}
		
		   \gothamset{block=transparent}
			\begin{verbatim}\gothamset{
				block=transparent}\end{verbatim}
		
		   \begin{block}{Default}
		     Block content.
		   \end{block}
		
		   \begin{alertblock}{Alert}
		     Block content.
		   \end{alertblock}
		
		   \begin{exampleblock}{Example}
		     Block content.
		   \end{exampleblock}

		\column{0.3\textwidth}
		
		   \gothamset{block=fill}
			\begin{verbatim}\gothamset{
				block=fill}\end{verbatim}
		
		   \begin{block}{Default}
		     Block content.
		   \end{block}
		
		   \begin{alertblock}{Alert}
		     Block content.
		   \end{alertblock}
		
		   \begin{exampleblock}{Example}
		     Block content.
		   \end{exampleblock}
		
		\end{columns}
	\end{frame}

	% FRAME
	\begin{frame}[fragile]{Color customization}
		The colors can be changed using:
		\begin{verbatim}
		\colorlet{colorPale}{gPaleYell} % BG in light/normal mode
		\colorlet{colorDark}{gDarkBlack} % FG in light/normal mode
		\colorlet{colorA}{gDarkTeal} % frametitle, standin.out,
		\colorlet{colorAreversed}{gLightTeal} % frametitle, standin.in,
		\colorlet{colorB}{gMidGrey} % gray BG : progress bar, blocks
		\colorlet{colorC}{gDeepYellOr} % progress bar
		\colorlet{colorD}{gLightOrange} % alert
		\colorlet{colorE}{gLightGreen} % example 
		\end{verbatim}
	\end{frame}


\subsection{Inner}

	% FRAME
	\begin{frame}[fragile]{Title page}
		\themename{} offers the possibility to adapt the title page layout (printed with \verb|\maketitle| or \verb|\titlepage|).
		This can be achieved using:

		\begin{verbatim}   \defbeamertemplate{title page}{your name}{your defintion}
			\gothamset{title page= your name}\end{verbatim}
		
		\themename{} also predefined several templates such as: 
		\verb$gotham normal | gotham splitvert | gotham dividedpic$ \verb$| gotham reversed$
	\end{frame}

	% FRAME
	\begin{frame}[fragile]{Table of contents}
		\themename{} come with the possibility to apply different style for your table of contents (ToC) page.
		You can define your own ToC style as it follows:
		\begin{verbatim}
			\defbeamertemplate{toc page}{your name}{your def}
			\gothamset{tocframe template= your name}
		\end{verbatim}
		Then, referring to this template using the frame option \verb|[toc]| in your presentation: 
		\begin{verbatim}
			\begin{frame}[toc]{Table of contents}
				\tableofcontents%[hideallsubsections]
			\end{frame }\end{verbatim}

		Or using one of the \themename{} predefined template, such as: \verb$gotham simple | gotham bullet$
	\end{frame}

	% FRAME
	\begin{frame}[fragile]{Sections}
		\themename{} provides a multiple options to tune sections (respectively \verb|part|, \verb|section|, \verb|subsection| and \verb|subsubsection|).
		Thus, using the setting controls:
		
		The section command \verb|\section{Elements}| from Beamer will appear very differently.
		The section page will appear or disappear thanks to: \verb$\gothamset{sectionframe default=<on|off>}$, while its layout (when appearing) is controlled by: 
		\begin{verbatim}
			\defbeamertemplate{part|sub|subsub|section frame}
				{your name}{your def}
			\gothamset{sectionframe template= your name}\end{verbatim}

		\themename{} predefined template are: \verb$gotham progressbar | gotham simple |$ \verb$gotham splitvert progressbar |$ \verb$gotham splitvert simple | gotham progressvert$
	\end{frame}

	% FRAME
	\begin{frame}[fragile]{Sections contents}
		After the section page, you can (de)activate a page with table of contents in the section using \verb$\gothamset{sectiontocframe default=<on|off>}$, and its layout is controlled by: 
		\begin{verbatim}
			\defbeamertemplate{toc subsection frame}{your name}{your def}
			\gothamset{sectionframe template= your name}
		\end{verbatim}

		\themename{} predefined template are: \verb$gotham simple | gotham bullet$
	\end{frame}

	% FRAME
	\begin{frame}[fragile, watermark]{Watermark}

		With \themename{} you can locally or globally add watermark to your slides by using:
		\begin{verbatim}  \defbeamertemplate{background}{watermark/your name}{your def}
		  \gothamset{watermark template= your name}\end{verbatim}
		
		Then, this watermark can be turn on locally using \verb|\begin{frame}[watermark]| or globally with \verb|\gothamset{watermark default= on}| .
 	\end{frame}

	% FRAME
	\begin{standinenv}
	\begin{frame}[fragile]{Standin}

		\themename{} comes with 2 environments/specials layouts named \verb|standin| and \verb|standout|.
		These specials layouts can be used to emphasize some content or last slide\textellipsis

		This layout can be turn on using \verb|\begin{frame}[standin]| or using the dedicated environment (\verb|\begin{standinenv}\begin{frame}...\end{frame}\end{standinenv}|).

		Note that the background can also be tuned using:
		\begin{verbatim}  \defbeamertemplate{background canvas}{standin/name}{your def}
		  \gothamset{standin template= name}\end{verbatim}
		
	\end{frame}
	\end{standinenv}

	% FRAME
	\begin{frame}[standout, watermark]{Standout}
		Here is an example of standout (working as standin), that can be combined with a watermark.

		Another difference, apart the obvious color change is the font size and series.
	\end{frame}


\subsection{Outer}
	
	{%
		\setbeamertemplate{frame footer}{My custom footer}
	% FRAME
	\begin{frame}[fragile]{Frame footer}
	   \themename{} defines a custom Beamer template to add a text to the footer. 
		It can be set via
	   \begin{verbatim}\setbeamertemplate{frame footer}{My custom footer}\end{verbatim}

		Even after redefining (or not) your frame footer template, you can locally remove it with the frame option \verb|\begin{frame}[nofooter]|.
	\end{frame}
	}

	\title[your shorttitle]{Gotham}
	\date[shortdate]{\today}
	\author[your shortauthor name]{Romain NOËL}
	% FRAME
	\begin{frame}[fragile, rotateFooter]{rotateFooter}
		The default footer from \themename{}, it displays the \verb|shortdate|, \verb|shorttitle| and \verb|shortauthor|.
		So by filling these fields in your document setup, you will see them appear in your footer:
		\begin{verbatim}   \title[your shorttitle]{Your title}
			\date[shortdate]{\today}
			\author[your shortauthor name]{John DOE} \end{verbatim}

		Since, we always need some extra space on some frames that would like to overlay a bit the footer, \themename{}'s footer offers also possibility to be put on side locally using \verb|\begin{frame}[rotateFooter]|, or globally with 
		\begin{verbatim} \gothamset{rotateFooter default=on} \end{verbatim}
		If it has set globally, it can be deactivated locally with the frame option \verb|\begin{frame}[norotateFooter]|.
	\end{frame}

	\title[]{Gotham}
	\date[]{\today}
	\renewcommand{\gothamRightFiligrane}{%
		\rotatebox{90}{gotham right filigrane pattern}
	}
	% FRAME
	\begin{frame}[edging, fragile]{Edging}
		\themename{} has two hook commands, \verb|\gothamRightFiligrane| and \verb|\gothamLeftFiligrane|, that can be redefined to customize what to display in the edgings (a.k.a. filigrane, a.k.a. sidebar).
		As example, one could do:
		\begin{verbatim}
			\renewcommand{\gothamRightFiligrane}{%
				\rotatebox{90}{gotham right filigrane pattern}
			}\end{verbatim}

		Then, to set if it should be displayed or not, globally \begin{verbatim} \gothamset{edging default=on} \end{verbatim}
		or locally with the frame option \verb|\begin{frame}[edging]| or \verb|\begin{frame}[noedging]|.
	\end{frame}

	% FRAME
	% \begin{nofootlineenv}
	\begin{frame}[fragile,noedging,nofooter]{Really wide contents}
		\begin{adjustwidth}{-2em}{-2em}
			If you want a really wide content in your frame, you can change the size of your margin (requires \verb|\usepackage{changepage}| in your preamble).
			You can also suppress the edging (\verb|[noedging]|) and footer (\verb|[nofooter]|) or even more radically footline (\verb|[nofootline]|).

			Here is an example combining them: 
			\begin{verbatim}
				\begin{frame}[noedging,nofootline]{extended frame}
					\begin{adjustwidth}{-2em}{-2em}% 2em extra to the left and 2em for right margin.
						wide content
					\end{adjustwidth}
				\end{frame }
			\end{verbatim}
		\end{adjustwidth}
	\end{frame}
	% \end{nofootlineenv}

	{%
	\renewcommand{\gothamInstituteLogoSquare}[1][4ex]{%
      \includegraphics[height=#1]{gotham-logo.pdf}
   }
	\logo{extra LOGO}
	% FRAME
	\begin{frame}[fragile]{Frametitle}
		\framesubtitle{with a subtitle}
		The frametile template brought by \themename{} is relatively classic: it supports \verb|\subframetitle| and frame continuation (with \verb|[allowframebreaks]|) through templates that can be tuned.
		Nevertheless, it the frametitle template also includes a hook for your institute logo in the top right corner, leaving the command \verb|\logo{}| free for your extra logos.
		
		So, one can have both logos using:
		\begin{verbatim}
			\renewcommand{\gothamInstituteLogoSquare}[1][4ex]{
				\includegraphics[height=#1]{gotham-logo.pdf}
			}
			\logo{extra LOGO}
		\end{verbatim}
	\end{frame}
	}

	\author[]{Romain NOËL}
	{%
		\gothamset{progressbar position=foot, numbering= totalframenumber}
	% FRAME
	\begin{frame}[fragile]{Numbering and progressbar}

		\themename{} theme can numbering your frames in the bottom right corner using different styles. 
		You can also decide to use a progression bar to indicate how much of your presentation remains.

		The setup of numbering and progression bar can be performed through:
		\begin{verbatim}
			\gothamset{numbering= totalframenumber, progressbar position=foot}
		\end{verbatim}

		Numbering available options are: \verb$none | framenumber | totalframenumber | appendixframenumber | pagenumber | totalpagenumber | circle$

		Progressbar position available options are: \verb$none | head | frametitle | foot | circlehead$
	\end{frame}
	}


\end{document}
%EoF
%
\section{UGE Theme}

	\begin{frame}
		\frametitle{Why this template ?}

		This template is here to help UGE Beamer users forgotten by the administration.
		\bigskip

		The template is available via the following links:\\
		\textbullet GitHub: %\\
		\url{https://github.com/KirmTwinty/uge-beamer}

		\textbullet Overleaf: %\\
		\url{https://www.overleaf.com/read/vxnjgfmyvccj}
		\bigskip

		Currently the main authors and maintainers of the template are:
		Romain NOËL and Thibaud TOULLIER.\newline
		We are always looking for any good wills to help us.
	\end{frame}


	\begin{frame}[fragile]
		\frametitle{Title Page}

		The UGE template is keeping Beamer primitives such as: \\
		\quad \verb#\institute# for affiliations, or \\
		\quad \verb#\titlegraphic# for logos on the title page, or \\
		\quad \verb#aspectratio=169# or \verb|43| for the beamer class option (with other ratios the layout was not setup).
		\bigskip
		
		The background can be changed manually using:
		\begin{verbatim} 
		{\usebackgroundtemplate{\includegraphics{YourBackground.pdf}}
		\begin{frame} \titlepage \end{frame}
		}
		\end{verbatim}
	\end{frame}


	\begin{frame}[fragile]
		\frametitle{Conserved Metropolis settings}

		The UGE style also respect primitives from Metropolis !

		So the pages for section can use a progress bar, the frame can numbered and use a progress bar in the footer or the header, the background can be light or dark...
		\bigskip

		All of these feature are accessible at anytime using:%
		\begin{verbatim}
			\metroset{
			   sectionpage= progressbar,
			   numbering=fraction,
			   progressbar= foot, 
			   block= fill,
			   background= light,
			}
		\end{verbatim}
	\end{frame}


	\begin{frame}[fragile]
		\frametitle{Dependencies}

		So the UGE theme cannot work without previously loading some dependencies:
		\begin{verbatim}
			% Include some custum modifications but quite handy.
			\usetheme{BeamerExtra}
			% Metropolis Theme.
			\usetheme{metropolis}
			% Change progress bar size and redefine some colors.
			\usetheme{MetropolisBeamerExtra}
		\end{verbatim}
	\end{frame}


	\begin{frame}[fragile, noedging]
		\frametitle{Normal slide}

		The default normal slide is including:
		\begin{itemize}
			\item a header (from Metropolis) with the UGE logo
			\item an extra logo: it can be change via the beamer command \verb|\logo{}|
			\item an email address in the footer: it can be disable by using in the preamble 
				\verb|\author[]{Your NAME\inst{1} ...}|
			\item a edging on left side (from UGE graphical charter). This can be turned off locally using \verb|\begin{frame}[noedging]| or globally with \verb|\boolfalse{defaultedging}|. If it is globally disabled it can be turned on locally using \verb|\begin{frame}[edging]|
		\end{itemize}
	\end{frame}


	\begin{frame}[fragile, rotateFooter, watermark]
		\frametitle{Extra frame options}

		\begin{itemize}
			\item \verb|\begin{frame}[rotateFooter]|
			\item a watermark on left side (from UGE graphical charter). This can be turned off locally using \verb|\begin{frame}[nowatermark]| or globally with \verb|\boolfalse{defaultWaterMark}|. If it is globally disabled it can be turned on locally using \verb|\begin{frame}[watermark]| .
			\item \verb|\renewcommand{\UGEwatermark}{otot}|
		\end{itemize}
	\end{frame}


	\begin{frame}[fragile, standin]
		\frametitle{Standout vs standin}

		From Metropolis theme a standout slide is available and was inspired by the UGE visit card Style.
		
		But also in addition the inverted color of the standout, called standin, was introducted.
		This standin style is used for section slide layout.
		
		Both standout and standin layout can be used on any slide using 
		\verb|\begin{frame}[standout]| and \verb|\begin{frame}[standin]|.

		From BeamerExtra the boolean \verb|sectionContent| (default is True) is used to print the content of the section after the section frame. 
		This can be turned of with \verb|\boolfalse{sectionContent}|.
	\end{frame}


	\begin{frame}[fragile]
		\frametitle{Sections options}

		\begin{itemize}
			\item \verb|\boolfalse{sectionContent}|
			\item \verb|\renewcommand{\secContentName}{toto}|
		\end{itemize}
	\end{frame}


	\begin{frame}[fragile]
		\frametitle{Lenghts}

		\begin{itemize}
			\item \verb|\setlength{footlineOffset}{Opt}|
			\item \verb|\setlength{footerOffset}{Opt}|
		\end{itemize}
	\end{frame}


	\begin{frame}
		\frametitle{Graphical guidelines not respected}

		By doing this UGE template, we also took some liberties about the graphical charter.
		Thus using the template you are accepting consequences.
		\bigskip
		
		Some elements of the graphical charter not respected:
		\begin{alertblock}{Alert}
			\begin{itemize}
			 	\item fonts because the official ones are not free and not easy to install
			 	\item title page since the official one is not convenient
			\end{itemize}
	   \end{alertblock}
	\end{frame}


	\begin{frame}[fragile]{Known Issues}
		\begin{alertblock}{Alert}
			\begin{itemize}
			 	%\item First compilation might fails.
					% But compile a second time and it should be fine.
				 	% This is due to 'numbered' section in the appendices that 
					% can not be numbered. FIXED !!
			 	\item Incompatibilities between \verb|thrm| and \verb|alert| (but this a Beamer issue).
			\end{itemize}
	   \end{alertblock}
	\end{frame}

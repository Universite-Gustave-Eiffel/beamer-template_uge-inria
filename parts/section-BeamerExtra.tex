\sectionpic{BeamerExtra}{\PathSO/UGE/logos/UGE-logo.pdf}
	
	\begin{frame}[fragile]{Backup slides}
		Sometimes, it is useful to add slides at the end of your presentation to
		refer to during audience questions.
		
		The best way to do this is to include the \verb|appendixnumberbeamer|
		package in your preamble and call \verb|\appendix| before your backup slides.
		
		\themename will automatically turn off slide numbering and progress bars for
		slides in the appendix.
	\end{frame}
	
	\begin{frame}[fragile, label=previouSlide]{Section With Pictures}
		As the following command indicates, it is possible to include pictures under a section title:
		\begin{verbatim}			
			\sectionpic{BeamerExtra}{\PathSO/UGE/logos/UGE.pdf}		
		\end{verbatim}
	\end{frame}

	\begin{frame}[fragile]{Section Content}
		The table of content after the section slide can be turn on or off with the command:
		\begin{verbatim}
			\boolfalse{sectionContent} % to turn off the table of content.
		\end{verbatim}
	
		From example, the table of content will be turn of for the next section (on the following slide).

		The name the content slide can also be modified thanks to the command:
		\begin{verbatim}
			\renewcommand{\secContentName}{def} % to redefine the title on section content.
		\end{verbatim}
	\end{frame}
	
	\renewcommand{\secContentName}{refined name}
	\boolfalse{sectionContent}
\section{Things good to know about Beamer (not BeamerExtra)}
	\booltrue{sectionContent}
	
	\begin{frame}[fragile]
		\frametitle{link to hidden slide}
		
		Thanks to Beamer, you can create buttons to just to hidden slides that are in the appendices
		\begin{verbatim}			
			\begin{frame}[label=<originalSlide>, noframenumbering]	
			\hyperlink{<hiddenSlide>}{\beamerbutton{NameOfButton}}
		\end{verbatim}
		
		Don't forget to create a button to come back also!
		
		For example :
		\hyperlink{previouSlide}{\beamerbutton{back to previous slide !}}
	\end{frame}
	
	\begin{frame}<all:1>[fragile]
		\frametitle{Print extended version}
		\framesubtitle{1}
		
		This slide will always be printed with 
		\begin{verbatim}			
			\begin{frame}<handout:1|beamer:1> or \begin{frame}<all:1>
		\end{verbatim}
		
		while the following code will never appears
		\begin{verbatim}			
			\begin{frame}<handout:0|beamer:0> or \begin{frame}<all:0>	
		\end{verbatim}
	\end{frame}
	
	\begin{frame}<handout:0|beamer:1>[fragile]
		\frametitle{Print extended version}
		\framesubtitle{2}
		
		This frame will appear with
		\begin{verbatim}			
			\documentclass[]{beamer}
			\begin{frame}<handout:0|beamer:1>
		\end{verbatim}
		
		BUT the following will not appear
		\begin{verbatim}			
				% requires handout to appear
				\documentclass[10pt]{beamer} 
				\begin{frame}<handout:1|beamer:0>
		\end{verbatim}
		
		BY opposition with \begin{verbatim} \documentclass[handout]{beamer}
		\end{verbatim}
	\end{frame}
			
	\begin{frame}<handout:1|beamer:0>[fragile]
		\frametitle{Print extended version}
		\framesubtitle{2}
		
		This frame will appear with
		\begin{verbatim}			
			\documentclass[handout]{beamer}
			\begin{frame}<handout:1|beamer:0>
		\end{verbatim}
		
		BUT the following will not appear
		\begin{verbatim}			
		    % requires to remove handout to appear
			\documentclass[handout]{beamer} 
			\begin{frame}<handout:0|beamer:1>
		\end{verbatim}
		
		BY opposition with \begin{verbatim} \documentclass[]{beamer}
		\end{verbatim}
	\end{frame}
	
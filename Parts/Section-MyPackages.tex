\section{Things brought by good practices in UserPackage}

	\begin{frame}
		\frametitle{nTheorem}

		The interactions beteween ntheorem and beamer has been solved.

		\begin{theorem}
			contenu...
		\end{theorem}

		\begin{proof}
			contenu...
		\end{proof}
	\end{frame}

	\begin{frame}
		\frametitle{Glossaries}

		Here is an example of acronym using glossaries: \gls{lbm}

		And an example of equation in colors, the Boltzmann equation:
		\begin{equation}
			{\color{blue} \partial_t \gls{BoltzDist}(x,\xi,t)} + {\color{red} \xi\cdot\partial_x \gls{BoltzDist}(x,\xi,t)} + {\color{green!60!black!80} g\cdot\partial_\xi f(x,\xi,t)}
			= {\color{yellow!80!orange!85!black} \Omega(\gls{BoltzDist},\gls{BoltzDist})}
		\end{equation}
		where $\gls{BoltzDist}$ is a variable defined through glossaries.

	\end{frame}

	\begin{frame}[fragile]
		\frametitle{Cite in the footnote}

		Some text that requires a reference\footfullciteBeamer{Knuth92},
		using the command\footnote{Please note that this command can be used side-by-side the command footnote.}
		\begin{verbatim}
				\footfullciteBeamer{<bibKey>}
		\end{verbatim}

		The command to cite in text mode is also possible with
		\begin{verbatim}
				\fullciteBeamer{<bibKey>}
		\end{verbatim}
		and giving \fullciteBeamer{ConcreteMath}
	\end{frame}

	\begin{frame}%[fragile]
		\frametitle{Video using media9/multimedia}

		\begin{columns}[c,onlytextwidth]
		\column{0.49\textwidth}
		
			The last but not least feature present is a macro simplifying the way to include videos in beamer presentation.

			It requires to define the OS on which the pdf will be red, thanks to the command:
			\begin{center}
				\ttfamily
				\textbackslash def\textbackslash OSvar\{linux\}
			\end{center}

			Then, the video can be included with the command:\\[0.25em]
			\begin{flushleft}
				\ttfamily
				\textbackslash includeVideo[\% no space\\
				\qquad width=7cm, height=5cm,%\\
				]\% Options \\
				\{\textbackslash PathS/video.mp4\}\% Video File\\
				\{\textbackslash includegraphics[width, height]\%\\
				\qquad	\{\textbackslash PathS/screenshot.png\}\%\\
				\}\% Poster image.
			\end{flushleft}

		\column{0.49\textwidth}
			Leading to the following result:

			\begin{figure}
				\centering
				\includeVideo[% no space
						width=0.75\linewidth, height=0.54\linewidth,
					]% Options
					{\PathS/Air_convec.mp4}% Video File
					{\includegraphics[width=0.75\linewidth,height=0.54\linewidth]%
						{\PathS/Thermal_convec.png}%
					}
				\caption{Example of embedded video using \texttt{\textbackslash includeVideo}.}
				\label{vid:includeVideo}
			\end{figure}
		\end{columns}
	\end{frame}

	\begin{frame}[fragile]{Warning about \texttt{\textbackslash includeVideo}.}
		\begin{alertblock}{Warning}
			Currently the method to include video using \texttt{\textbackslash includeVideo} is no longer work on Windows because \texttt{flash player} is no longer supported.
		\end{alertblock}
		Up to now, this solution is still working on Linux with \texttt{okular}, \texttt{poppler} and \texttt{phonon-backend-vlc} installed.

		A possible workaround is to used external player rather than embedded solutions.
		An example of external inclusion on Windows with \texttt{multimedia} package is:
		\begin{verbatim}
			\movie[externalviewer]%
	   {\includegraphics[width=\textheight]%
	      {figures/image.jpg}%
	   }{video_beamer.mp4}
		\end{verbatim}
	\end{frame}


	\newcounter{r}
	\newcommand{\escalar}[1]{
	\setcounter{r}{#1 * #1 * #1}
	}
	%
	\newcounter{m}
	\setcounter{m}{0}
	\newcounter{mc}
	%
	\begin{frame}[fragile]{Video using animate}

		\vspace{-2em}
		\begin{columns}[c,onlytextwidth]
		\column{0.49\textwidth}

			Another solution which is use a animation from a stack of image:
			\begin{verbatim}
				\animategraphics[autoplay, loop,
				   	width=\textwidth, controls
					]%
				{<frame rate>}}{Images/Image_}{0}{99}
			\end{verbatim}
			or for more complex drawing:
			\begin{verbatim}
				\begin{animateinline}[<opt>]{<rate>}
		   ... typeset material ...
				\newframe[<frame rate>]
		   ... typeset material ...
				\newframe
		   ...
				\end{animateinline}
			\end{verbatim}
		\column{0.49\textwidth}
			\begin{figure}
				\centering
				\begin{animateinline}[loop, poster = first, controls, palindrome]{2}
					\whiledo{\them < 21}{
		    \begin{tikzpicture}[scale=0.95]
			   \draw[red,thick,<->] (-1,1) parabola bend (0,0) (2.1,4.41)
			       node[below right] {$y=x^2$};
			   \draw[loosely dotted] (-1,0) grid (4,4);
			   %\path[use as bounding box] (-2,-1) rectangle (5,5);
			   \draw[->] (-0.2,0) -- (4.25,0) node[right] {$x$};
			   \draw[->] (0,-0.25) -- (0,4.25) node[above] {$y$};
			   \foreach \x/\xtext in {1/1, 2/2, 3/3}
			   \draw[shift={(\x,0)}] (0pt,2pt) -- (0pt,-2pt) node[below] {$\xtext$};
			   \foreach \y/\ytext in {1/1, 2/2, 3/3, 4/4}
			   \draw[shift={(0,\y)}] (2pt,0pt) -- (-2pt,0pt) node[left] {$\ytext$};
						%
			   \setcounter{mc}{\value{m}*\value{m}}
			   \shade[top color=blue,bottom color=gray!50]
			       (0,0) parabola (0.1*\them,0.01*\themc) |- (0,0);
			   \escalar{\them}
			   \draw (3cm,2pt) node[above]
			       {$\displaystyle\int\limits_0^{\them/10} \!\!x^2\mathrm{d}x =
			           \displaystyle\frac{\ther}{3000}$};
			   \draw[fill=orange,color=orange] (0.1*\them,0.01*\themc) circle (0.5pt);
		   \end{tikzpicture}
				%
				\stepcounter{m}
				\ifthenelse{\them < 21}{
					\newframe
				}{
				\end{animateinline}\relax % BREAK
				} } % END \whiledo...
				\caption{Example of embedded video using \texttt{\textbackslash animategraphics}.}
				\label{vid:animate}
      	\end{figure}
     	\end{columns}
	\end{frame}

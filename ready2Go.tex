% !TEX TS-program = pdflatex
% !TeX encoding = UTF-8
% !TeX spellcheck = en_GB
%=======================================================================
%== Template for LaTeX presentation ====================================
%=======================================================================
\providecommand{\PathP}{./parts/} 		% Path to Parts folders
\providecommand{\PathR}{./resources/} 	% Path to Resources folders
\providecommand{\PathSO}{./sources/} 	% Path to Sources folders
\providecommand{\PathS}{./supplies/} 	% Path to Supplies folders
\providecommand{\BuildTrigger}{buildnew} % Trigger to build or not standalone parts.
\providecommand{\langue}{french}		% english, french

\makeatletter
	\ifx\c@levelstanda\undefined 	% to create a newcounter if it doesn't exists.
		\newcounter{levelstanda}
		\def\input@path{ {\PathSO}{\PathSO/gotham}{\PathSO/UGE/}{\PathSO/Inria/} } % to be able to find package in source folders.
	\else
		{}% it already exist
	\fi
\makeatother


\documentclass[notheorems, noamsthm, aspectratio=169, 10pt]{beamer}
	% use this instead for 16:9 aspect ratio: [aspectratio=169]
	% supported aspect ratios  1610  169 149 54 43 (default) 32
	%
	% The option "handout, " allows to the print the hidden slides made by <handout:1|beamer:0>[noframenumbering].
	%
	% the option "notes=only, " to print only the notes !
	\usepackage[l2tabu, orthodox]{nag}	% Help to prevent buggus with warnings

%-=-=-=-=-=-=-=-=-=-=-=-=-=-=-=-=-=-=-=-=-=-=-=-=
%        DEFAULT PACKAGES
%-=-=-=-=-=-=-=-=-=-=-=-=-=-=-=-=-=-=-=-=-=-=-=-=
	\usepackage[utf8]{inputenc}
	\usepackage[T1]{fontenc}


%-=-=-=-=-=-=-=-=-=-=-=-=-=-=-=-=-=-=-=-=-=-=-=-=
%        BEAMER OPTIONS
%-=-=-=-=-=-=-=-=-=-=-=-=-=-=-=-=-=-=-=-=-=-=-=-=
	% Gotham Theme
	\usetheme{gotham}
	\usetheme{gothamUGE}
		\UgeSet{unofficial}% {official,unofficial}
	% \usetheme{gothamInria}
	% 	\InriaSet{old1}% {RF,old1,old2,old3}
	% \usetheme{gothamCEA}
	% 	\CEASet{official}% {official}

		% Apply all Gotham setting.
		\gothamset{
			% format frametitle=titlecase,
			% format title=titlecase, format section=titlecase,
			% sectionframe default=on, sectiontocframe default=on,
			numbering= totalframenumber, %none, framenumber,
			progressbar position= foot, %none, head, frametitle, foot
			background= light, %dark, %light, %transparent,
		}


%-=-=-=-=-=-=-=-=-=-=-=-=-=-=-=-=-=-=-=-=-=-=-=-=
%        LOADING MY PACKAGES
%-=-=-=-=-=-=-=-=-=-=-=-=-=-=-=-=-=-=-=-=-=-=-=-=
		\newcommand{\Pathbasebib}{\PathR/Presentation.bib}
		\newcommand{\PathSymbolsEntries}{\PathR/UsersSymbolsAcronyms}%.bib
		\def\OSvar{linux} % windows, mac, linux
	%%%% 		UsersPackagesCommands.tex
%%%%		version 0.1.1 du 01/11/2023
%%%% 		Author: Romain NOËL
%%%%		email: romain.noel@univ-eiffel.fr
%%%%--------------------------------------------%%%%

%%%%---- OS & FONT     -------------------------%%%%
%%%%---- POLICE     ----------------------------%%%%
%%%%---- TEXT FORMATTING -----------------------%%%%
	% \usepackage{enumitem}
	% \usepackage{ulem}

	%%%% Personal %%%%
	\makeatletter
		\newcommand{\ie}{\@ifstar{\emph{i.e.\@}}{\emph{id est}\ }}
		\newcommand{\eg}{\@ifstar{\emph{e.g.\@}}{\emph{exempli gratia}\ }}
		\newcommand{\cad}{\@ifstar{c.-\`a.-d.\@}{c'est-à-dire}\ }
		\newcommand{\cf}{\@ifstar{\emph{cf.\@}}{\emph{confer}\ }}
	\makeatother

	\ExplSyntaxOn
		% Create a new command to fake Small Caps by scaling uppercase with one mandatory argument.
		\DeclareDocumentCommand{\fakesc}{ m }{
			\textapperance_smallcaps:n { #1 }
		}

		%–– user-adjustable scaling parameters
		\tl_new:N \l__fauxsc_hscale_dim  \tl_set:Nn \l__fauxsc_hscale_dim {.83}
		\tl_new:N \l__fauxsc_vscale_dim  \tl_set:Nn \l__fauxsc_vscale_dim {.72}%.72

		\tl_new:N \l__textapperance_textsmallcaps_input_tl% Initialize string variable
		\cs_new_protected:Npn \textapperance_smallcaps:n #1 {
			% Fully expand contents and store them in the variable
			\tl_set:Nx \l__textapperance_textsmallcaps_input_tl { #1 }
			
			% Replace everything thats NOT a capital letter NEITHER punctuation by \textsmaller[2]{\uppercase{<match>}}
			\regex_replace_all:nnN { ([^A-Z\ \,\.\;\:\!\?\'\"\(\)\[\]\{\}0-9]+) }{ 
				% { \c{relscale}{0.75}\cB\{\c{uppercase}\cB\{\0\cE\}\cE\}} } 
				\c{scalebox}{
					\c{l__fauxsc_hscale_dim}}[\c{l__fauxsc_vscale_dim}]{
					\cB\{\c{uppercase}\cB\{\0\cE\}\cE\}
				}
			} \l__textapperance_textsmallcaps_input_tl 
			\tl_use:N \l__textapperance_textsmallcaps_input_tl % print the result
		}

		% Adapt some language element to the command \langue.
		% Test: does \langue exist?
		\cs_if_exist:cTF { langue }{
			% Yes, it exists.  
			% Now compare its expansion to the string "french".
			\str_if_eq:VnTF \langue { french }{
				% ⟨true⟩ branch so its adapt some commands
				\renewcommand{\partContentName}{Contenu\ de\ la\ partie}
				\renewcommand{\secContentName}{Contenu\ de\ la\ section}
				\renewcommand{\subsecContentName}{Contenu\ de\ la\ sous-section}
			}{
				% ⟨false⟩ branch (exists but isn’t “french”) so do nothing
			}
		}{
			% Command doesn’t exist
			\typeout{\string\langue~does~not~exist.}
		}
	\ExplSyntaxOff

	\newcommand{\famName}[1]{\fakesc{#1}}%\textsc


%%%%---- SPACING & JUSTIFICATION ---------------%%%%
%%%%---- LAYOUT & MARGINS ----------------------%%%%
	\usepackage{changepage}
   \usepackage{appendixnumberbeamer} % Manage frame numbering in appendixes in beamer


%%%%---- FOOT PAGE & HEAD ----------------------%%%%
%%%%---- TABLE of CONTENT & FIGURES ------------%%%%
%%%%---- PARTS & CHAPTERS ----------------------%%%%
%%%%---- WATERMARKS ----------------------------%%%%
%%%%---- COVER PAGE ----------------------------%%%%
%%%%---- IMAGES --------------------------------%%%%
	%\usepackage{placeins} % provide FloatBarrier
	%\usepackage[absolute,overlay]{textpos} % for positioning text or floats at absolute coords.
	\usepackage{float} % finer control of float and new float env
	\usepackage{subcaption} % multiple figures
		%\renewcommand{\thesubfigure}{{\normalfont\alph{subfigure}}}%


%%%%---- DRAWING & GRAPHS ----------------------%%%%
	\usepackage{import}
	\usepackage{tikz} %
	\usepackage{pgfplots} %
		\pgfplotsset{compat=newest} %
		\pdfsuppresswarningpagegroup=1


%%%%---- TABLES & BOXES ------------------------%%%%
	% \usepackage{tcolorbox} % colored box with a lot of flexibility
	% 	\tcbuselibrary{skins,theorems} % extra for pseudo(code)
	\usepackage{tabularray} % Typeset tabulars and arrays (contains equivalent of longtable, booktabs and dcolumn at least)
		\UseTblrLibrary{booktabs} % to load extra commands from booktabs
		% \UseTblrLibrary{diagbox} % for cells with a diagbox
		% \UseTblrLibrary{varwidth} % for cell with variable width
		% \UseTblrLibrary{amsmath} % to improve with +array, +matrix, +bmatrix, +Bmatrix, +pmatrix, +vmatrix, +Vmatrix and +cases
		% \UseTblrLibrary{siunitx} % for cell using SI formatting


%%%%---- SYMBOLS -------------------------------%%%%
%%%%---- MATHS & PHYSICS -----------------------%%%%
	\usepackage{amsmath, amsfonts, amssymb} % American Math Standard
	\usepackage{mathtools} % extension of AMS with tools
	\usepackage{mleftright} % better left and right bracket management
	\usepackage{derivative} % standard and customable derivative operators
	\usepackage{siunitx} % International System units
	% \usepackage{tensorsop} % work in progress ^^
	\usepackage{amsthm, thmtools} % replacement of ntheorem
		\declaretheorem{theorem}
		% Define a theorem style that uses Beamer's 'block' environment
		\renewenvironment{theorem}[1][]{%
			\refstepcounter{theorem}% Increment the theorem counter
			\begin{block}{\thmname{Theorem}~\thmnumber{\thetheorem}\if\relax\detokenize{#1}\relax\else~(#1)\fi:}%
		}{%
			\end{block}%
		}
	% \usepackage{empheq} % emphasize equations
	\usepackage[only,llbracket,rrbracket]{stmaryrd}
	% \usepackage{bm}


%%%%---- THEOREMS ------------------------------%%%%
%%%%---- CHEMISTRY & BIO -----------------------%%%%
%%%%---- CODING, LINKS & OTHERS ----------------%%%%
	% \usepackage{listings} % to include piece of codes
	\usepackage{pseudo} % for flexible pseudocode
		\floatstyle{ruled} % create a env for algorithms with Float
		\newfloat{algorithm}{htbp}{alg}
		\floatname{algorithm}{Algorithm}
	\usepackage{cleveref}%
	\usepackage{standalone} 	% Other include compilable alone.


%%%%---- INDEX & GLOSSARIES --------------------%%%%
	\usepackage[record, automake, section=section]{glossaries-extra}
		%Newglossaries
		\newglossary*{acronym}{Abbreviations}
		\newglossary*{notation}{Notation}
		\newglossary*{symbol}{Nomenclature}

		% new keys must be defined before \GlsXtrLoadResources
		\glsaddstoragekey{unit}{}{\glsentryunit}

		% define style of the abbreviation in the text
		\setabbreviationstyle[acronym]{long-short}
		\setabbreviationstyle[foreignabbreviation]{long-short-user}
		\setabbreviationstyle[inline]{long-short}	%
		\setabbreviationstyle[footer]{footnote} 	%


		% \newcommand{\AtEndPreambPerso}{
		\GlsXtrLoadResources[
			src={\PathSymbolsEntries}, % entries in symbols.bib
			%selection=all,
			%save-locations=false,% location list not needed
			entry-type-aliases={main=entry, notation=entry, foreignabbreviation=abbreviation},
			field-aliases={
				foreignlong=user1,
				nativelong=long
			},
			type=acronym,
			match={entrytype=(acronym|abbreviation)},
			category={same as original entry}% requires bib2gls v1.4+
			%sort
		]

		\GlsXtrLoadResources[
			src={\PathSymbolsEntries}, % entries in symbols.bib
			%selection=all,
			%save-locations=false,% location list not needed
			entry-type-aliases={main=entry, notation=entry, foreignabbreviation=abbreviation},
			type=symbol,
			match={entrytype=symbol},
			sort-field={category},% sort @symbol entries by category field
			symbol-sort-fallback=description,% sort @symbol entries by name field
			identical-sort-action=name,% sort by description if sort value identical
			sort={letter-nocase},
			% symbol-sort-fallback=name,% sort @symbol entries by name field
			% identical-sort-action=description,% sort by description if sort value identical
		]
		% }
		\glsdisablehyper


%%%%---- BIBLIOGRAPHY --------------------------%%%%
	\usepackage[%citestyle=numeric,  	% style de citation : alphabetic, numeric, 		authoryear,
		%bibstyle=numeric, 	% style de la biblio : alphabetic, authoryear, numeric,
		style=numeric,			% style gloabel + apa, ieee, ams : style predefinit par certains organisme
		sorting=none,
		mcite=true,				% collapsing  multiple  citations into one
		subentry,				% have entry that referes to another (not with authoryear)
		labelnumber, 			%
		citetracker=true,		% To track citations
		% backref=true,			% afficher les pages ou les citations sont dans la biblio
		url=false,
		giveninits=true,   	% Initials for first name
		maxcitenames=3,			% nb max de noms lors de \cite, audela "et al."
		maxbibnames=100,		% nb max de noms dans la biblio, audela "et al."
		defernumbers=true,		% ne pas charger toute la base de donnees, uniquement le necessaire
		backend=biber			% bibtex, biber
	]{biblatex}

		% \AtEveryBibitem{\clearfield{url}}
		\renewcommand{\mkbibnamefamily}[1]{\textsc{#1}}

		% Let us now load the bib files.
		%% \addbibresource[datatype=bibtex]{\Pathbasebib}
		% requires etoolbox to loop from a CSV list. (etoolbox is loaded implicitly by ...)
		%\addbibresource[datatype=bibtex]{\Pathbasebib}
		\newcommand{\mylistDo}[1]{ \addbibresource[datatype=bibtex]{#1} }%
		%\forcsvlist{\mylistDo}{\expandafter\Pathbasebib}
		\expandafter\forcsvlist\expandafter\mylistDo\expandafter{\Pathbasebib}

		% Create a \defbibentrysetlabel command in order to redefine the printing of mcite manually
		\makeatletter
			% warning the file called \blxmset@bibfile@name will be
			% overwritten without warning
			\def\blxmset@bibfile@name{\jobname -msets.bib}
			\newwrite\blxmset@bibfile
			\immediate\openout\blxmset@bibfile=\blxmset@bibfile@name
			\immediate\write\blxmset@bibfile{%
				@comment{auxiliary file for \string\defbibentrysetlabel}^^J%
				@comment{This file may safely be deleted.
					It will be recreated as required.}}

			\AtEndDocument{%
				\closeout\blxmset@bibfile}

			\newrobustcmd*{\defbibentrysetlabel}[3]{%
				\@bsphack
				\immediate\write\blxmset@bibfile{%
					@set{#1, entryset = {\unexpanded{#3}}, %
						shorthand = {\unexpanded{#2}},}%
				}%
				\nocite{#1}%
				\@esphack}

			\addbibresource{\blxmset@bibfile@name}
		\makeatother

		\renewcommand{\entrysetpunct}{.\newline}

		% Change the footnote style with letters to avoid confusion.
		\renewcommand{\thefootnote}{\alph{footnote}}
		% Change the style of the printing biblio to let appear the numbers like in the footnote
		\setbeamertemplate{bibliography item}{\insertbiblabel}

		% Define the new cite command using the footnote
		\DeclareCiteCommand{\footfullciteBeamer}
		{\usebibmacro{prenote}}
		{%
			\renewcommand{\thefootnote}{\arabic{footnote}}% Switch to footnote with numbers
			\footnotemark[\thefield{labelnumber}]% Add the mark corresponding to the number entry%
			\footnotetext[\thefield{labelnumber}]{%  Add the footnote text with same number entry.
				% \printfield{labelnumber}
				\printnames{labelname}% The name
				\setunit{\printdelim{nametitledelim}}% separator
				\printfield[citetitle]{labeltitle}% The title
				\setunit{\addperiod\space}% separator
				\printfield{year}% The year
			}%
			\renewcommand{\thefootnote}{\alph{footnote}}% Switch back to footnote with letters.
		}
		{\multicitedelim}
		{\usebibmacro{postnote}}

		%
		\DeclareCiteCommand{\fullciteBeamer}
		{\usebibmacro{prenote}}
		{\printnames{labelname}%
			\setunit{\printdelim{nametitledelim}}%
			\printfield[citetitle]{labeltitle}%
			\setunit{\addperiod\space}%
			\printfield{year}%
		}
		{\multicitedelim}
		{\usebibmacro{postnote}}


%%%%---- MULTIMEDIA ---------------------------%%%%
	% from float package
		\floatstyle{plain} % create a env for algorithms with Float
		\newfloat{video}{H}{vid}
		\floatname{video}{Video}
	\usepackage{animate} % Create embedded animation from images on all OS and viewer !

		% Os variables for the macros
		%\def\OSvar{linux} % windows, mac, linux
		\def\OSlinux{linux}
		\def\OSwindows{windows}
		\def\OSmac{mac}
		\providecommand{\includeVideo}[3][]{The command includeVideo is not correctly defined for your OS.}

	%%%%%%%%%%%%%%%%%%%%%%%%%%%%%%%%%%%%%%%%
	% To convert Video with right codec !!!!
	%%%%%%%%%%%%%%%%%%%%%%%%%%%%%%%%%%%%%%%%
	% ffmpeg -i Bubble_light.avi -vf scale="trunc(iw/2)*2:trunc(ih/2)*2" -c:v libx264 -profile:v high -pix_fmt yuv420p -g 30 -r 30 Bubble_light.mp4

		% Load the useful package for linux
		\ifx\OSvar\OSlinux
			\usepackage{multimedia}

			\renewcommand{\includeVideo}[3][]{%
				\movie[ % On linux with okular ++ poppler and phonon-backend-vlc installed
					showcontrols=true, %
					%poster, %
					#1% some options like size
				]%
				{#3}%
				{#2}
			}
		\fi
		% for windows
		\ifx\OSvar\OSwindows
			\usepackage{embedvideo}
			% \usepackage{media9}

			% \renewcommand{\includeVideo}[3][]{%
			% 	\includemedia[% % Windows AcrobatReader >9.1
			% 		%	activate=pagevisible,%
			% 		%   activate=onclick, %this is default
			% 		%	deactivate=pageclose,%
			% 		addresource=#2,%
			% 		flashvars={%
			% 			src=#2 % same path as in addresource!
			% 			%	&autoPlay=true % default: false; if =true, automatically starts playback after activation (see option ‘activation)’
			% 			%	&loop=true % if loop=true, media is played in a loop
			% 			%	&controlBarAutoHideTimeout=0 %  time span before auto-hide
			% 		},%
			% 		#1% some options like size%
			% 	]{#3}{StrobeMediaPlayback.swf}% %{SlideShow.swf} %{APlayer.swf} %{StrobeMediaPlayback.swf} %{VPlayer.swf}
			% }% end of the new command
		\fi


%%%%---- TRANSLATION ---------------------------%%%%
%%%%---- VARIABLES -----------------------------%%%%
%%%%---- MACROS --------------------------------%%%%
%%%%---- HYPHENATIONS & ACRONYMS --------------%%%%

	%%%% 		UsersPackagesCommands.tex			%%%%
%%%%		version 0.0.1 du 12/12/2020			%%%%
%%%% 		Author: Romain NOËL					%%%%
%%%%		email: romain.noel@univ-eiffel.fr			%%%%
%%%%--------------------------------------------%%%%

%%%%---- OS & FONT     -------------------------%%%%
%%%%---- POLICE     ----------------------------%%%%
%%%%---- TEXT FORMATTING -----------------------%%%%
%%%%---- SPACING & JUSTIFICATION ---------------%%%%
%%%%---- LAYOUT & MARGINS ----------------------%%%%
%%%%---- FOOT PAGE & HEAD ----------------------%%%%
%%%%---- TABLE of CONTENT & FIGURES ------------%%%%
%%%%---- PARTS & CHAPTERS ----------------------%%%%
%%%%---- WATERMARKS ----------------------------%%%%
%%%%---- COVER PAGE ----------------------------%%%%
%%%%---- IMAGES & FLOAT ------------------------%%%%
%%%%---- DRAWING & GRAPHS ----------------------%%%%
%%%%---- TABLES & BOXES ------------------------%%%%
%%%%---- SYMBOLS -------------------------------%%%%
%%%%---- MATHS & PHYSICS -----------------------%%%%
		\newcommand{\Rea}{\ensuremath{\mathbb{R} }} 
		\newcommand{\Com}{\ensuremath{\mathbb{C} }} 
		\newcommand{\diff}{\ensuremath{\operatorname{d}\!}} 	% 
		\newcommand{\diffe}{\ensuremath{\operatorname{d}\!}}
		\newcommand{\diverg}{\mathop{\textrm{div}}\nolimits} 		% 
		\newcommand{\gradi}{\mathop{\textrm{grad}}\nolimits} 	%
		\newcommand{\rota}{\mathop{\textrm{rot}}\nolimits} 		%
		\DeclareMathOperator{\Tr}{Tr}
		\DeclareMathOperator{\erf}{erf}
		\DeclareMathOperator{\Diag}{diag}
		\newcommand{\adher}[1]{\overline{#1}}
		\newcommand{\norm}[2][]{\ensuremath{\Vert{#2}\Vert_{#1}}}
		\newcommand{\absval}[2][]{\ensuremath{\vert{#2}\vert_{#1}}}
		\newcommand{\prodSca}[2]{\ensuremath{\left\langle{#1}\middle|{#2}\right\rangle }}
		\newcommand{\scalar}{\ensuremath{\cdot}}
		\newcommand{\dyadic}[3][]{\ensuremath{ {#2}\otimes^{#1}{#3} }}
		\newcommand{\contra}[3][]{\ensuremath{ {#2}{\,\overset{#1}{\cvdots}}\,{#3} }}
		\newcommand{\doublecontra}[2]{\ensuremath{ {#1}\, :\,{#2} }}
		\newcommand{\transpo}[1]{\ensuremath{ {#1}^{T} }}
		\newcommand{\bigOh}[1]{\ensuremath{ \mathcal{O}\!\left({#1}\right) }}
		\newcommand{\smalloh}[1]{\ensuremath{ \omicron\!\left({#1}\right) }}
		\newcommand{\convo}[2]{\ensuremath{ {#1}\star{#2} }}
		
		\makeatletter
			\newcommand{\@parenth}[1]{\left(#1\right)}
			\newcommand{\@indexparenth}[2][]{_{#1}\left(#2\right)}
			\newcommand{\@indexparenthStar}[2][]{_{#1}\, #2}
			\newcommand{\@subsupscriparenth}[3][]{_{#1}^{#2}\left(#3\right)}
			\newcommand{\@subsupscriparenthStar}[3][]{_{#1}^{#2}\, #3}
			\newcommand{\divergN}{\nabla\@ifstar{\@divergNStar}{\@divergNNostar}}
				\newcommand{\@divergNNostar}[2][]{_{#1}\cdot\left(#2\right)}
				\newcommand{\@divergNStar}[2][]{_{#1}\cdot #2}
			\newcommand{\gradN}{\nabla\@ifstar{\@indexparenthStar}{\@indexparenth}}
			\newcommand{\LaplaN}{\mathop{}\!\nabla^2\@ifstar{\@indexparenthStar}{\@indexparenth}}
			\newcommand{\DalembN}{\mathop{}\!\mathbin\Box\@ifstar{\@indexparenthStar}{\@indexparenth}}
			\newcommand{\rotN}{\nabla\@ifstar{\@rotNStar}{\@rotNNostar}}
				\newcommand{\@rotNNostar}[2][]{_{#1}\wedge\left(#2\right)}
				\newcommand{\@rotNStar}[2][]{_{#1}\wedge #2}
			\newcommand{\divergO}{\diverg\@ifstar{\@indexparenthStar}{\@indexparenth}}
			\newcommand{\gradO}{\gradi\@ifstar{\@indexparenthStar}{\@indexparenth}}
			\newcommand{\rotO}{\rota\@ifstar{\@indexparenthStar}{\@indexparenth}}
			\newcommand{\LaplaO}{\mathop{}\!\mathbin\bigtriangleup\@ifstar{\@indexparenthStar}{\@indexparenth}}
			\newcommand{\DalembO}{\mathop{}\!\mathbin\Box\@ifstar{\@indexparenthStar}{\@indexparenth}}
			\newcommand{\gradNp}{\nabla\@ifstar{\@subsupscriparenthStar}{\@subsupscriparenth}}
		\makeatother
		
%%%%---- THEOREMS ------------------------------%%%%
%%%%---- CHEMISTRY & BIO -----------------------%%%%
%%%%---- CODING --------------------------------%%%%
%%%%---- INCLUDE OTHER FILES -------------------%%%%
%%%%---- LaTeX PROGRAMMING  --------------------%%%%
%%%%---- TESTING & EXTRA -----------------------%%%%
%%%%---- LINKS & REFERENCES --------------------%%%%
%%%%---- INDEX & GLOSSARIES --------------------%%%%
%%%%---- BIBLIOGRAPHY --------------------------%%%%
%%%%---- TRANSLATION ---------------------------%%%%
%%%%---- VARIABLES -----------------------------%%%%
%%%%---- MACROS --------------------------------%%%%

%%%%---- HYPHENATIONS & ACRONYMS --------------%%%%
  \hyphenation{an-ti-cons-ti-tu-tion-el-le-ment bal-lon di-men-sion-ne-ment}
  \hyphenation{é-qua-tion}


	% To see notes !
	%\setbeameroption{show notes} % un-comment to see the notes in
	%\usepackage{pgfpages}
	% See notes
	%\setbeameroption{show notes on second screen=right} % Requires \usepackage{pgfpages} and pympress or pdfpc


%-=-=-=-=-=-=-=-=-=-=-=-=-=-=-=-=-=-=-=-=-=-=-=-=
%	PRESENTATION INFORMATION
%-=-=-=-=-=-=-=-=-=-=-=-=-=-=-=-=-=-=-=-=-=-=-=-=

	\title[]{Univ-Eiffel Template long long longl long long longl long long longl long long longl}
	\subtitle{A modern beamer theme based}
	\date[]{\today}
	\author[romain.noel@univ-eiffel.fr]{
		Romain \famName{Noël}\inst{1}\inst{2}
		\and John \famName{Doe}\inst{3}
	}
	\institute[Univ. Eiffel]{
		\inst{1} Université Gustave Eiffel
		\and \inst{2} Inria Rennes
		\and \inst{3} An Awesome Company
	}
	\titlegraphic{%
		\hspace*{0.52\textwidth}\vspace{1.6mm}%
		\includegraphics[height=0.08\paperheight]{\PathSO/Inria/logos/inr_logo_rouge.pdf}%
	}

	\logo{\includegraphics[width=1.5cm]{\PathSO/Inria/logos/inr_logo_rouge.pdf}}
	\addtobeamertemplate{logo}{\vspace{0.76\paperheight}}{} % Top Right
	%\addtobeamertemplate{logo}{}{\hspace{0.885\paperwidth}} % Bottom Left
	%\addtobeamertemplate{logo}{\vspace{0.72\paperheight}}{\hspace{0.885\paperwidth}} % Top Left

	\hypersetup{
		pdfpagemode = FullScreen, % Display the presentation in fullscreen at launch.
		pdfauthor 	= {R. NOEL}, %
		pdfproducer = {LaTeX-Beamer} %
	}


%-=-=-=-=-=-=-=-=-=-=-=-=-=-=-=-=-=-=-=-=-=-=-=-=
%	DOCUMENT
%-=-=-=-=-=-=-=-=-=-=-=-=-=-=-=-=-=-=-=-=-=-=-=-=
\begin{document}
\stepcounter{levelstanda}

%%%%%%%%%%%%%%%%%%%%%%
%%%  FRONT-MATTER  %%%
%%%%%%%%%%%%%%%%%%%%%%
	% FRAME
	\begin{frame}[plain,noframenumbering]
		\titlepage
		\note{Everything you want}
	\end{frame}

	% FRAME
	\begin{frame}[toc]{Outline}
		\tableofcontents[hideallsubsections]
	\end{frame}


%%%%%%%%%%%%%%%%%%%%%
%%%  MAIN-MATTER  %%%
%%%%%%%%%%%%%%%%%%%%%
\section{name your section}
% !TEX TS-program = pdflatex
% !TeX encoding = UTF-8
% !TeX spellcheck = en_GB
%===========================================================================
%== template for LaTeX presentation ========================================
%===========================================================================
\providecommand{\PathP}{./}  % Path to Parts folders
\providecommand{\PathR}{../resources/}  % Path to Resources folders
\providecommand{\PathSO}{../sources/} 	% Path to Sources folders
\providecommand{\PathS}{../supplies/} 	% Path to Supplies folders
\providecommand{\BuildTrigger}{buildnew} % Trigger to build or not standalone parts.
\providecommand{\lang}{english} % english, french

\makeatletter
	\def\input@path{{\PathSO}{\PathSO/gotham}}  % to be able to find package in resource folders.
	\ifx\c@levelstanda\undefined 	% to create a newcounter if it doesn't exists.
		\newcounter{levelstanda}
	\else
		{}% it already exist
	\fi
\makeatother


\documentclass[notheorems, noamsthm, aspectratio=169, 10pt]{beamer}
	% use this instead for 16:9 aspect ratio: [aspectratio=169]
	% supported aspect ratios  1610  169 149 54 43 (default) 32
	%
	% The option "handout, " allows to the print the hidden slides made by <handout:1|beamer:0>[noframenumbering].
	%
	% the option "notes=only, " to print only the notes !
	\usepackage[l2tabu, orthodox]{nag}	% Help to prevent buggus with warnings
	\usepackage[utf8]{inputenc}
	\usepackage[T1]{fontenc}


%-=-=-=-=-=-=-=-=-=-=-=-=-=-=-=-=-=-=-=-=-=-=-=-=
%        BEAMER OPTIONS
%-=-=-=-=-=-=-=-=-=-=-=-=-=-=-=-=-=-=-=-=-=-=-=-=
	% Gotham Theme
	\usetheme{gotham}

		% Apply all Gotham setting.
		\gothamset{
			% titleformat=%regular, smallcaps, allsmallcaps, allcaps
			% format frametitle=titlecase,
			% format title=titlecase, format section=titlecase,
			% sectionframe default=on, sectiontocframe default=on,
			numbering= totalframenumber, %none, framenumber,
			progressbar position= foot, %none, head, frametitle, foot
			background= light, %dark, %light, %transparent,
		}


%-=-=-=-=-=-=-=-=-=-=-=-=-=-=-=-=-=-=-=-=-=-=-=-=
%        LOADING MY PACKAGES
%-=-=-=-=-=-=-=-=-=-=-=-=-=-=-=-=-=-=-=-=-=-=-=-=
		\newcommand{\Pathbasebib}{\PathR/Presentation.bib}
		\newcommand{\PathSymbolsEntries}{\PathR/UsersSymbolsAcronyms}%.bib
		\def\OSvar{linux} % windows, mac, linux
	%%%% 		UsersPackagesCommands.tex
%%%%		version 0.1.1 du 01/11/2023
%%%% 		Author: Romain NOËL
%%%%		email: romain.noel@univ-eiffel.fr
%%%%--------------------------------------------%%%%

%%%%---- OS & FONT     -------------------------%%%%
%%%%---- POLICE     ----------------------------%%%%
%%%%---- TEXT FORMATTING -----------------------%%%%
	% \usepackage{enumitem}
	% \usepackage{ulem}

	%%%% Personal %%%%
	\makeatletter
		\newcommand{\ie}{\@ifstar{\emph{i.e.\@}}{\emph{id est}\ }}
		\newcommand{\eg}{\@ifstar{\emph{e.g.\@}}{\emph{exempli gratia}\ }}
		\newcommand{\cad}{\@ifstar{c.-\`a.-d.\@}{c'est-à-dire}\ }
		\newcommand{\cf}{\@ifstar{\emph{cf.\@}}{\emph{confer}\ }}
	\makeatother

	\ExplSyntaxOn
		% Create a new command to fake Small Caps by scaling uppercase with one mandatory argument.
		\DeclareDocumentCommand{\fakesc}{ m }{
			\textapperance_smallcaps:n { #1 }
		}

		%–– user-adjustable scaling parameters
		\tl_new:N \l__fauxsc_hscale_dim  \tl_set:Nn \l__fauxsc_hscale_dim {.83}
		\tl_new:N \l__fauxsc_vscale_dim  \tl_set:Nn \l__fauxsc_vscale_dim {.72}%.72

		\tl_new:N \l__textapperance_textsmallcaps_input_tl% Initialize string variable
		\cs_new_protected:Npn \textapperance_smallcaps:n #1 {
			% Fully expand contents and store them in the variable
			\tl_set:Nx \l__textapperance_textsmallcaps_input_tl { #1 }
			
			% Replace everything thats NOT a capital letter NEITHER punctuation by \textsmaller[2]{\uppercase{<match>}}
			\regex_replace_all:nnN { ([^A-Z\ \,\.\;\:\!\?\'\"\(\)\[\]\{\}0-9]+) }{ 
				% { \c{relscale}{0.75}\cB\{\c{uppercase}\cB\{\0\cE\}\cE\}} } 
				\c{scalebox}{
					\c{l__fauxsc_hscale_dim}}[\c{l__fauxsc_vscale_dim}]{
					\cB\{\c{uppercase}\cB\{\0\cE\}\cE\}
				}
			} \l__textapperance_textsmallcaps_input_tl 
			\tl_use:N \l__textapperance_textsmallcaps_input_tl % print the result
		}

		% Adapt some language element to the command \langue.
		% Test: does \langue exist?
		\cs_if_exist:cTF { langue }{
			% Yes, it exists.  
			% Now compare its expansion to the string "french".
			\str_if_eq:VnTF \langue { french }{
				% ⟨true⟩ branch so its adapt some commands
				\renewcommand{\partContentName}{Contenu\ de\ la\ partie}
				\renewcommand{\secContentName}{Contenu\ de\ la\ section}
				\renewcommand{\subsecContentName}{Contenu\ de\ la\ sous-section}
			}{
				% ⟨false⟩ branch (exists but isn’t “french”) so do nothing
			}
		}{
			% Command doesn’t exist
			\typeout{\string\langue~does~not~exist.}
		}
	\ExplSyntaxOff

	\newcommand{\famName}[1]{\fakesc{#1}}%\textsc


%%%%---- SPACING & JUSTIFICATION ---------------%%%%
%%%%---- LAYOUT & MARGINS ----------------------%%%%
	\usepackage{changepage}
   \usepackage{appendixnumberbeamer} % Manage frame numbering in appendixes in beamer


%%%%---- FOOT PAGE & HEAD ----------------------%%%%
%%%%---- TABLE of CONTENT & FIGURES ------------%%%%
%%%%---- PARTS & CHAPTERS ----------------------%%%%
%%%%---- WATERMARKS ----------------------------%%%%
%%%%---- COVER PAGE ----------------------------%%%%
%%%%---- IMAGES --------------------------------%%%%
	%\usepackage{placeins} % provide FloatBarrier
	%\usepackage[absolute,overlay]{textpos} % for positioning text or floats at absolute coords.
	\usepackage{float} % finer control of float and new float env
	\usepackage{subcaption} % multiple figures
		%\renewcommand{\thesubfigure}{{\normalfont\alph{subfigure}}}%


%%%%---- DRAWING & GRAPHS ----------------------%%%%
	\usepackage{import}
	\usepackage{tikz} %
	\usepackage{pgfplots} %
		\pgfplotsset{compat=newest} %
		\pdfsuppresswarningpagegroup=1


%%%%---- TABLES & BOXES ------------------------%%%%
	% \usepackage{tcolorbox} % colored box with a lot of flexibility
	% 	\tcbuselibrary{skins,theorems} % extra for pseudo(code)
	\usepackage{tabularray} % Typeset tabulars and arrays (contains equivalent of longtable, booktabs and dcolumn at least)
		\UseTblrLibrary{booktabs} % to load extra commands from booktabs
		% \UseTblrLibrary{diagbox} % for cells with a diagbox
		% \UseTblrLibrary{varwidth} % for cell with variable width
		% \UseTblrLibrary{amsmath} % to improve with +array, +matrix, +bmatrix, +Bmatrix, +pmatrix, +vmatrix, +Vmatrix and +cases
		% \UseTblrLibrary{siunitx} % for cell using SI formatting


%%%%---- SYMBOLS -------------------------------%%%%
%%%%---- MATHS & PHYSICS -----------------------%%%%
	\usepackage{amsmath, amsfonts, amssymb} % American Math Standard
	\usepackage{mathtools} % extension of AMS with tools
	\usepackage{mleftright} % better left and right bracket management
	\usepackage{derivative} % standard and customable derivative operators
	\usepackage{siunitx} % International System units
	% \usepackage{tensorsop} % work in progress ^^
	\usepackage{amsthm, thmtools} % replacement of ntheorem
		\declaretheorem{theorem}
		% Define a theorem style that uses Beamer's 'block' environment
		\renewenvironment{theorem}[1][]{%
			\refstepcounter{theorem}% Increment the theorem counter
			\begin{block}{\thmname{Theorem}~\thmnumber{\thetheorem}\if\relax\detokenize{#1}\relax\else~(#1)\fi:}%
		}{%
			\end{block}%
		}
	% \usepackage{empheq} % emphasize equations
	\usepackage[only,llbracket,rrbracket]{stmaryrd}
	% \usepackage{bm}


%%%%---- THEOREMS ------------------------------%%%%
%%%%---- CHEMISTRY & BIO -----------------------%%%%
%%%%---- CODING, LINKS & OTHERS ----------------%%%%
	% \usepackage{listings} % to include piece of codes
	\usepackage{pseudo} % for flexible pseudocode
		\floatstyle{ruled} % create a env for algorithms with Float
		\newfloat{algorithm}{htbp}{alg}
		\floatname{algorithm}{Algorithm}
	\usepackage{cleveref}%
	\usepackage{standalone} 	% Other include compilable alone.


%%%%---- INDEX & GLOSSARIES --------------------%%%%
	\usepackage[record, automake, section=section]{glossaries-extra}
		%Newglossaries
		\newglossary*{acronym}{Abbreviations}
		\newglossary*{notation}{Notation}
		\newglossary*{symbol}{Nomenclature}

		% new keys must be defined before \GlsXtrLoadResources
		\glsaddstoragekey{unit}{}{\glsentryunit}

		% define style of the abbreviation in the text
		\setabbreviationstyle[acronym]{long-short}
		\setabbreviationstyle[foreignabbreviation]{long-short-user}
		\setabbreviationstyle[inline]{long-short}	%
		\setabbreviationstyle[footer]{footnote} 	%


		% \newcommand{\AtEndPreambPerso}{
		\GlsXtrLoadResources[
			src={\PathSymbolsEntries}, % entries in symbols.bib
			%selection=all,
			%save-locations=false,% location list not needed
			entry-type-aliases={main=entry, notation=entry, foreignabbreviation=abbreviation},
			field-aliases={
				foreignlong=user1,
				nativelong=long
			},
			type=acronym,
			match={entrytype=(acronym|abbreviation)},
			category={same as original entry}% requires bib2gls v1.4+
			%sort
		]

		\GlsXtrLoadResources[
			src={\PathSymbolsEntries}, % entries in symbols.bib
			%selection=all,
			%save-locations=false,% location list not needed
			entry-type-aliases={main=entry, notation=entry, foreignabbreviation=abbreviation},
			type=symbol,
			match={entrytype=symbol},
			sort-field={category},% sort @symbol entries by category field
			symbol-sort-fallback=description,% sort @symbol entries by name field
			identical-sort-action=name,% sort by description if sort value identical
			sort={letter-nocase},
			% symbol-sort-fallback=name,% sort @symbol entries by name field
			% identical-sort-action=description,% sort by description if sort value identical
		]
		% }
		\glsdisablehyper


%%%%---- BIBLIOGRAPHY --------------------------%%%%
	\usepackage[%citestyle=numeric,  	% style de citation : alphabetic, numeric, 		authoryear,
		%bibstyle=numeric, 	% style de la biblio : alphabetic, authoryear, numeric,
		style=numeric,			% style gloabel + apa, ieee, ams : style predefinit par certains organisme
		sorting=none,
		mcite=true,				% collapsing  multiple  citations into one
		subentry,				% have entry that referes to another (not with authoryear)
		labelnumber, 			%
		citetracker=true,		% To track citations
		% backref=true,			% afficher les pages ou les citations sont dans la biblio
		url=false,
		giveninits=true,   	% Initials for first name
		maxcitenames=3,			% nb max de noms lors de \cite, audela "et al."
		maxbibnames=100,		% nb max de noms dans la biblio, audela "et al."
		defernumbers=true,		% ne pas charger toute la base de donnees, uniquement le necessaire
		backend=biber			% bibtex, biber
	]{biblatex}

		% \AtEveryBibitem{\clearfield{url}}
		\renewcommand{\mkbibnamefamily}[1]{\textsc{#1}}

		% Let us now load the bib files.
		%% \addbibresource[datatype=bibtex]{\Pathbasebib}
		% requires etoolbox to loop from a CSV list. (etoolbox is loaded implicitly by ...)
		%\addbibresource[datatype=bibtex]{\Pathbasebib}
		\newcommand{\mylistDo}[1]{ \addbibresource[datatype=bibtex]{#1} }%
		%\forcsvlist{\mylistDo}{\expandafter\Pathbasebib}
		\expandafter\forcsvlist\expandafter\mylistDo\expandafter{\Pathbasebib}

		% Create a \defbibentrysetlabel command in order to redefine the printing of mcite manually
		\makeatletter
			% warning the file called \blxmset@bibfile@name will be
			% overwritten without warning
			\def\blxmset@bibfile@name{\jobname -msets.bib}
			\newwrite\blxmset@bibfile
			\immediate\openout\blxmset@bibfile=\blxmset@bibfile@name
			\immediate\write\blxmset@bibfile{%
				@comment{auxiliary file for \string\defbibentrysetlabel}^^J%
				@comment{This file may safely be deleted.
					It will be recreated as required.}}

			\AtEndDocument{%
				\closeout\blxmset@bibfile}

			\newrobustcmd*{\defbibentrysetlabel}[3]{%
				\@bsphack
				\immediate\write\blxmset@bibfile{%
					@set{#1, entryset = {\unexpanded{#3}}, %
						shorthand = {\unexpanded{#2}},}%
				}%
				\nocite{#1}%
				\@esphack}

			\addbibresource{\blxmset@bibfile@name}
		\makeatother

		\renewcommand{\entrysetpunct}{.\newline}

		% Change the footnote style with letters to avoid confusion.
		\renewcommand{\thefootnote}{\alph{footnote}}
		% Change the style of the printing biblio to let appear the numbers like in the footnote
		\setbeamertemplate{bibliography item}{\insertbiblabel}

		% Define the new cite command using the footnote
		\DeclareCiteCommand{\footfullciteBeamer}
		{\usebibmacro{prenote}}
		{%
			\renewcommand{\thefootnote}{\arabic{footnote}}% Switch to footnote with numbers
			\footnotemark[\thefield{labelnumber}]% Add the mark corresponding to the number entry%
			\footnotetext[\thefield{labelnumber}]{%  Add the footnote text with same number entry.
				% \printfield{labelnumber}
				\printnames{labelname}% The name
				\setunit{\printdelim{nametitledelim}}% separator
				\printfield[citetitle]{labeltitle}% The title
				\setunit{\addperiod\space}% separator
				\printfield{year}% The year
			}%
			\renewcommand{\thefootnote}{\alph{footnote}}% Switch back to footnote with letters.
		}
		{\multicitedelim}
		{\usebibmacro{postnote}}

		%
		\DeclareCiteCommand{\fullciteBeamer}
		{\usebibmacro{prenote}}
		{\printnames{labelname}%
			\setunit{\printdelim{nametitledelim}}%
			\printfield[citetitle]{labeltitle}%
			\setunit{\addperiod\space}%
			\printfield{year}%
		}
		{\multicitedelim}
		{\usebibmacro{postnote}}


%%%%---- MULTIMEDIA ---------------------------%%%%
	% from float package
		\floatstyle{plain} % create a env for algorithms with Float
		\newfloat{video}{H}{vid}
		\floatname{video}{Video}
	\usepackage{animate} % Create embedded animation from images on all OS and viewer !

		% Os variables for the macros
		%\def\OSvar{linux} % windows, mac, linux
		\def\OSlinux{linux}
		\def\OSwindows{windows}
		\def\OSmac{mac}
		\providecommand{\includeVideo}[3][]{The command includeVideo is not correctly defined for your OS.}

	%%%%%%%%%%%%%%%%%%%%%%%%%%%%%%%%%%%%%%%%
	% To convert Video with right codec !!!!
	%%%%%%%%%%%%%%%%%%%%%%%%%%%%%%%%%%%%%%%%
	% ffmpeg -i Bubble_light.avi -vf scale="trunc(iw/2)*2:trunc(ih/2)*2" -c:v libx264 -profile:v high -pix_fmt yuv420p -g 30 -r 30 Bubble_light.mp4

		% Load the useful package for linux
		\ifx\OSvar\OSlinux
			\usepackage{multimedia}

			\renewcommand{\includeVideo}[3][]{%
				\movie[ % On linux with okular ++ poppler and phonon-backend-vlc installed
					showcontrols=true, %
					%poster, %
					#1% some options like size
				]%
				{#3}%
				{#2}
			}
		\fi
		% for windows
		\ifx\OSvar\OSwindows
			\usepackage{embedvideo}
			% \usepackage{media9}

			% \renewcommand{\includeVideo}[3][]{%
			% 	\includemedia[% % Windows AcrobatReader >9.1
			% 		%	activate=pagevisible,%
			% 		%   activate=onclick, %this is default
			% 		%	deactivate=pageclose,%
			% 		addresource=#2,%
			% 		flashvars={%
			% 			src=#2 % same path as in addresource!
			% 			%	&autoPlay=true % default: false; if =true, automatically starts playback after activation (see option ‘activation)’
			% 			%	&loop=true % if loop=true, media is played in a loop
			% 			%	&controlBarAutoHideTimeout=0 %  time span before auto-hide
			% 		},%
			% 		#1% some options like size%
			% 	]{#3}{StrobeMediaPlayback.swf}% %{SlideShow.swf} %{APlayer.swf} %{StrobeMediaPlayback.swf} %{VPlayer.swf}
			% }% end of the new command
		\fi


%%%%---- TRANSLATION ---------------------------%%%%
%%%%---- VARIABLES -----------------------------%%%%
%%%%---- MACROS --------------------------------%%%%
%%%%---- HYPHENATIONS & ACRONYMS --------------%%%%

	%%%% 		UsersPackagesCommands.tex			%%%%
%%%%		version 0.0.1 du 12/12/2020			%%%%
%%%% 		Author: Romain NOËL					%%%%
%%%%		email: romain.noel@univ-eiffel.fr			%%%%
%%%%--------------------------------------------%%%%

%%%%---- OS & FONT     -------------------------%%%%
%%%%---- POLICE     ----------------------------%%%%
%%%%---- TEXT FORMATTING -----------------------%%%%
%%%%---- SPACING & JUSTIFICATION ---------------%%%%
%%%%---- LAYOUT & MARGINS ----------------------%%%%
%%%%---- FOOT PAGE & HEAD ----------------------%%%%
%%%%---- TABLE of CONTENT & FIGURES ------------%%%%
%%%%---- PARTS & CHAPTERS ----------------------%%%%
%%%%---- WATERMARKS ----------------------------%%%%
%%%%---- COVER PAGE ----------------------------%%%%
%%%%---- IMAGES & FLOAT ------------------------%%%%
%%%%---- DRAWING & GRAPHS ----------------------%%%%
%%%%---- TABLES & BOXES ------------------------%%%%
%%%%---- SYMBOLS -------------------------------%%%%
%%%%---- MATHS & PHYSICS -----------------------%%%%
		\newcommand{\Rea}{\ensuremath{\mathbb{R} }} 
		\newcommand{\Com}{\ensuremath{\mathbb{C} }} 
		\newcommand{\diff}{\ensuremath{\operatorname{d}\!}} 	% 
		\newcommand{\diffe}{\ensuremath{\operatorname{d}\!}}
		\newcommand{\diverg}{\mathop{\textrm{div}}\nolimits} 		% 
		\newcommand{\gradi}{\mathop{\textrm{grad}}\nolimits} 	%
		\newcommand{\rota}{\mathop{\textrm{rot}}\nolimits} 		%
		\DeclareMathOperator{\Tr}{Tr}
		\DeclareMathOperator{\erf}{erf}
		\DeclareMathOperator{\Diag}{diag}
		\newcommand{\adher}[1]{\overline{#1}}
		\newcommand{\norm}[2][]{\ensuremath{\Vert{#2}\Vert_{#1}}}
		\newcommand{\absval}[2][]{\ensuremath{\vert{#2}\vert_{#1}}}
		\newcommand{\prodSca}[2]{\ensuremath{\left\langle{#1}\middle|{#2}\right\rangle }}
		\newcommand{\scalar}{\ensuremath{\cdot}}
		\newcommand{\dyadic}[3][]{\ensuremath{ {#2}\otimes^{#1}{#3} }}
		\newcommand{\contra}[3][]{\ensuremath{ {#2}{\,\overset{#1}{\cvdots}}\,{#3} }}
		\newcommand{\doublecontra}[2]{\ensuremath{ {#1}\, :\,{#2} }}
		\newcommand{\transpo}[1]{\ensuremath{ {#1}^{T} }}
		\newcommand{\bigOh}[1]{\ensuremath{ \mathcal{O}\!\left({#1}\right) }}
		\newcommand{\smalloh}[1]{\ensuremath{ \omicron\!\left({#1}\right) }}
		\newcommand{\convo}[2]{\ensuremath{ {#1}\star{#2} }}
		
		\makeatletter
			\newcommand{\@parenth}[1]{\left(#1\right)}
			\newcommand{\@indexparenth}[2][]{_{#1}\left(#2\right)}
			\newcommand{\@indexparenthStar}[2][]{_{#1}\, #2}
			\newcommand{\@subsupscriparenth}[3][]{_{#1}^{#2}\left(#3\right)}
			\newcommand{\@subsupscriparenthStar}[3][]{_{#1}^{#2}\, #3}
			\newcommand{\divergN}{\nabla\@ifstar{\@divergNStar}{\@divergNNostar}}
				\newcommand{\@divergNNostar}[2][]{_{#1}\cdot\left(#2\right)}
				\newcommand{\@divergNStar}[2][]{_{#1}\cdot #2}
			\newcommand{\gradN}{\nabla\@ifstar{\@indexparenthStar}{\@indexparenth}}
			\newcommand{\LaplaN}{\mathop{}\!\nabla^2\@ifstar{\@indexparenthStar}{\@indexparenth}}
			\newcommand{\DalembN}{\mathop{}\!\mathbin\Box\@ifstar{\@indexparenthStar}{\@indexparenth}}
			\newcommand{\rotN}{\nabla\@ifstar{\@rotNStar}{\@rotNNostar}}
				\newcommand{\@rotNNostar}[2][]{_{#1}\wedge\left(#2\right)}
				\newcommand{\@rotNStar}[2][]{_{#1}\wedge #2}
			\newcommand{\divergO}{\diverg\@ifstar{\@indexparenthStar}{\@indexparenth}}
			\newcommand{\gradO}{\gradi\@ifstar{\@indexparenthStar}{\@indexparenth}}
			\newcommand{\rotO}{\rota\@ifstar{\@indexparenthStar}{\@indexparenth}}
			\newcommand{\LaplaO}{\mathop{}\!\mathbin\bigtriangleup\@ifstar{\@indexparenthStar}{\@indexparenth}}
			\newcommand{\DalembO}{\mathop{}\!\mathbin\Box\@ifstar{\@indexparenthStar}{\@indexparenth}}
			\newcommand{\gradNp}{\nabla\@ifstar{\@subsupscriparenthStar}{\@subsupscriparenth}}
		\makeatother
		
%%%%---- THEOREMS ------------------------------%%%%
%%%%---- CHEMISTRY & BIO -----------------------%%%%
%%%%---- CODING --------------------------------%%%%
%%%%---- INCLUDE OTHER FILES -------------------%%%%
%%%%---- LaTeX PROGRAMMING  --------------------%%%%
%%%%---- TESTING & EXTRA -----------------------%%%%
%%%%---- LINKS & REFERENCES --------------------%%%%
%%%%---- INDEX & GLOSSARIES --------------------%%%%
%%%%---- BIBLIOGRAPHY --------------------------%%%%
%%%%---- TRANSLATION ---------------------------%%%%
%%%%---- VARIABLES -----------------------------%%%%
%%%%---- MACROS --------------------------------%%%%

%%%%---- HYPHENATIONS & ACRONYMS --------------%%%%
  \hyphenation{an-ti-cons-ti-tu-tion-el-le-ment bal-lon di-men-sion-ne-ment}
  \hyphenation{é-qua-tion}


	% To see notes !
	%\setbeameroption{show notes} % un-comment to see the notes in
	%\usepackage{pgfpages}
	% See notes
	%\setbeameroption{show notes on second screen=right} % Requires \usepackage{pgfpages} and pympress or pdfpc


\begin{document} %
\stepcounter{levelstanda}

%%%  MAIN-MATTER  %%%
%%%%%%%%%%%%%%%%%%%%

%\section{A new section}

	% FRAME
	\begin{frame}{title of your frame}
		your first frame
	\end{frame}


\addtocounter{levelstanda}{-1}
\end{document}


	% FRAME
	\begin{frame}[standout, plain]
		\centering
		Thank you for your attention !
		
		---

		Do you have any question ?
	\end{frame}


%%%%%%%%%%%%%%%%%%%%%
%%%  BACK-MATTER  %%%
%%%%%%%%%%%%%%%%%%$%%
	% \begin{frame}[noframenumbering, allowframebreaks]{References}
	% 	\printbibliography[heading=none]
	% 	% \bibliography{demo}
	% 	% \bibliographystyle{abbrv}
	% \end{frame}


%%  APPENDIX  %%
%%%%%%%%%%%%%%%%
\appendix
% \miniframesoff % to deactivate the navigation bar

	% FRAME
	\begin{frame}[standout, plain]
		\centering
		Backup !
	\end{frame}

	% % FRAME
	% \begin{frame}{title}
	% 	toto
	% \end{frame}

\addtocounter{levelstanda}{-1}
\end{document}
